\documentclass[titlepage]{article}
\usepackage[utf8]{inputenc}
\usepackage[T1]{fontenc}
\usepackage{listings}
\usepackage{xcolor}
\usepackage{tabularx}
\usepackage{colortbl}

\title{Rapport MT10 - TP2 : Autour du RSA}
\author{Martin Schneider, Océane Bordeau}
\date{10 mai 2022}

\setlength{\parindent}{0pt}
\definecolor{codegreen}{rgb}{0,0.6,0}
\definecolor{codegray}{rgb}{0.5,0.5,0.5}
\definecolor{gray}{rgb}{0.8,0.8,0.8}
\definecolor{codepurple}{rgb}{0.58,0,0.82}
\definecolor{codeblue}{rgb}{0,0,255}
\definecolor{backcolour}{rgb}{0.95,0.95,0.92}

\lstdefinestyle{mystyle}{ 
    commentstyle=\color{magenta},
    keywordstyle=\color{codeblue},
    numberstyle=\tiny\color{codegray},
    stringstyle=\color{codepurple},
    basicstyle=\ttfamily\footnotesize,
    breakatwhitespace=false,         
    breaklines=true,                 
    captionpos=b,                    
    keepspaces=true,                                  
    showspaces=false,                
    showstringspaces=false,
    showtabs=false,                  
    tabsize=2
}

\lstset{style=mystyle}

\begin{document}
    \maketitle
    \tableofcontents
    \pagebreak

    \section{Préliminaire : \texttt{SageMath} et les nombres entiers}
    \textbf{Division euclidienne :}
    On utilise \texttt{divmod(x,y)} pour la division euclidienne de x par y, qui renvoie un couple $(div, mod)$ tel que :
    \[div*y + mod = x\]

    \begin{tabularx}{12cm}{|p{0.60cm}|X|}
        \hline
        \rowcolor{gray}
        \texttt{In}
        & 
        \texttt{divmod(15,4)}
        \\
        \hline
        \texttt{Out}
        &
        \texttt{(3,3)}
        \\
        \hline
    \end{tabularx}
    \bigbreak\bigbreak

    \textbf{Algorithme d'Euclide :}
    On utilise \texttt{gcd(a,b)} pour l'algorithme d'Euclide, qui renvoie le plus grand diviseur commun (PGCD) de a et b. \bigbreak

    \begin{tabularx}{12cm}{|p{0.60cm}|X|}
        \hline
        \rowcolor{gray}
        \texttt{In}
        & 
        \texttt{gcd(15,20)}
        \\
        \hline
        \texttt{Out}
        &
        \texttt{5}
        \\
        \hline
    \end{tabularx}
    \bigbreak\bigbreak

    \textbf{Algorithme d'Euclide étendu :}
    On utilise \texttt{xgcd(a,b)} pour l'algorithme d'Euclide étendu de a et b, qui renvoie un triple $(g, s, t)$ tel que :
    \[g = s*a + t*b = PGCD(a,b)\]

    \begin{tabularx}{12cm}{|p{0.60cm}|X|}
        \hline
        \rowcolor{gray}
        \texttt{In}
        & 
        \texttt{xgcd(15,20)}
        \\
        \hline
        \texttt{Out}
        &
        \texttt{(5,-1,-1)}
        \\
        \hline
    \end{tabularx}
    \bigbreak\bigbreak

    \textbf{Nombres premiers :} \bigbreak
    
    On utilise \texttt{is\_prime(x)} qui retourne \texttt{True} si le nombre est premier, \texttt{False} sinon. \bigbreak

    \begin{tabularx}{12cm}{|p{0.60cm}|X|}
        \hline
        \rowcolor{gray}
        \texttt{In}
        & 
        \texttt{is\_prime(15)}
        \\
        \hline
        \texttt{Out}
        &
        \texttt{False}
        \\
        \hline
    \end{tabularx}
    \bigbreak

    On utilise \texttt{prime\_range(x)} qui retourne la liste des nombres premiers strictement inférieurs à \texttt{x}. \bigbreak

    \begin{tabularx}{12cm}{|p{0.60cm}|X|}
        \hline
        \rowcolor{gray}
        \texttt{In}
        & 
        \texttt{prime\_range(15)}
        \\
        \hline
        \texttt{Out}
        &
        \texttt{[2, 3, 5, 7, 11, 13]}
        \\
        \hline
    \end{tabularx}
    \bigbreak

    On utilise \texttt{next\_prime(x)} qui retourne le premier nombre premier strictement supérieur à \texttt{x}. \bigbreak

    \begin{tabularx}{12cm}{|p{0.60cm}|X|}
        \hline
        \rowcolor{gray}
        \texttt{In}
        & 
        \texttt{next\_prime(17)}
        \\
        \hline
        \texttt{Out}
        &
        \texttt{19}
        \\
        \hline
    \end{tabularx}
    \bigbreak

    On utilise \texttt{factor(x)} qui retourne la décomposition de \texttt{x} en facteurs de nombres premiers. \bigbreak

    \begin{tabularx}{12cm}{|p{0.60cm}|X|}
        \hline
        \rowcolor{gray}
        \texttt{In}
        & 
        \texttt{factor(150)}
        \\
        \hline
        \texttt{Out}
        &
        \texttt{2 * 3 * 5$^ 2$ }
        \\
        \hline
    \end{tabularx}
    \bigbreak

    On utilise \texttt{prime\_pi(x)} qui compte combien de nombres premiers sont inférieurs où égaux à \texttt{x}. \bigbreak

    \begin{tabularx}{12cm}{|p{0.60cm}|X|}
        \hline
        \rowcolor{gray}
        \texttt{In}
        & 
        \texttt{prime\_pi(x)}
        \\
        \hline
        \texttt{Out}
        &
        \texttt{6}
        \\
        \hline
    \end{tabularx}
    \bigbreak

    \section{Nombres premiers}
    \subsection{Sur la répartition des nombres premiers}

    \subsection{Nombres de Fermat}

    \subsection{Nombres de Mersenne}

    \subsection{Un test de primalité pour les nombres de Mersenne}

    \section{Algorithmes d'exponentiation}
    \subsection{Naïf itératif et naïf récursif}

    \subsection{Dichotomique itératif et dichotomique récursif}

    \subsection{Algorithme d'exponentiation modulaire}

\end{document}
